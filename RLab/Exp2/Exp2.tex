\documentclass{article}\usepackage[]{graphicx}\usepackage[]{xcolor}
% maxwidth is the original width if it is less than linewidth
% otherwise use linewidth (to make sure the graphics do not exceed the margin)
\makeatletter
\def\maxwidth{ %
  \ifdim\Gin@nat@width>\linewidth
    \linewidth
  \else
    \Gin@nat@width
  \fi
}
\makeatother

\definecolor{fgcolor}{rgb}{0.345, 0.345, 0.345}
\newcommand{\hlnum}[1]{\textcolor[rgb]{0.686,0.059,0.569}{#1}}%
\newcommand{\hlsng}[1]{\textcolor[rgb]{0.192,0.494,0.8}{#1}}%
\newcommand{\hlcom}[1]{\textcolor[rgb]{0.678,0.584,0.686}{\textit{#1}}}%
\newcommand{\hlopt}[1]{\textcolor[rgb]{0,0,0}{#1}}%
\newcommand{\hldef}[1]{\textcolor[rgb]{0.345,0.345,0.345}{#1}}%
\newcommand{\hlkwa}[1]{\textcolor[rgb]{0.161,0.373,0.58}{\textbf{#1}}}%
\newcommand{\hlkwb}[1]{\textcolor[rgb]{0.69,0.353,0.396}{#1}}%
\newcommand{\hlkwc}[1]{\textcolor[rgb]{0.333,0.667,0.333}{#1}}%
\newcommand{\hlkwd}[1]{\textcolor[rgb]{0.737,0.353,0.396}{\textbf{#1}}}%
\let\hlipl\hlkwb

\usepackage{framed}
\makeatletter
\newenvironment{kframe}{%
 \def\at@end@of@kframe{}%
 \ifinner\ifhmode%
  \def\at@end@of@kframe{\end{minipage}}%
  \begin{minipage}{\columnwidth}%
 \fi\fi%
 \def\FrameCommand##1{\hskip\@totalleftmargin \hskip-\fboxsep
 \colorbox{shadecolor}{##1}\hskip-\fboxsep
     % There is no \\@totalrightmargin, so:
     \hskip-\linewidth \hskip-\@totalleftmargin \hskip\columnwidth}%
 \MakeFramed {\advance\hsize-\width
   \@totalleftmargin\z@ \linewidth\hsize
   \@setminipage}}%
 {\par\unskip\endMakeFramed%
 \at@end@of@kframe}
\makeatother

\definecolor{shadecolor}{rgb}{.97, .97, .97}
\definecolor{messagecolor}{rgb}{0, 0, 0}
\definecolor{warningcolor}{rgb}{1, 0, 1}
\definecolor{errorcolor}{rgb}{1, 0, 0}
\newenvironment{knitrout}{}{} % an empty environment to be redefined in TeX

\usepackage{alltt}
 \date{}
\title{Statistics for Engineers (MAT2001)- Lab  Experiment-II:  Correlation and Regression}
\IfFileExists{upquote.sty}{\usepackage{upquote}}{}
\begin{document}
\maketitle
 % \tableofcontents

\section{Karl Pearson’s Coefficient of Correlation}
\subsection{Scatter Diagram}
Problem: \newline
Illustrate the relationship between the average age versus the time spent in the
library, by using scatterplot.
\begin{table}[h]
\begin{tabular}{lccc}
AGE
GROUP & REPRESENTATIVE
AGE & HOURS SPEND IN THE LOCAL
LIBRARY &    \\
10-19 & 15  & 302.38  &    \\
20-29 & 25  & 193.63  &  \\
30-39 & 35  & 185.46  &   \\
40-49 & 45  & 198.49  &    \\
50-59 & 55  & 224.30  &    \\
60-69 & 65  & 288.71  &  
\end{tabular}
\end{table}
\begin{knitrout}
\definecolor{shadecolor}{rgb}{0.969, 0.969, 0.969}\color{fgcolor}\begin{kframe}
\begin{alltt}
\hldef{x} \hlkwb{<-} \hlkwd{c}\hldef{(}\hlnum{15}\hldef{,}\hlnum{25}\hldef{,}\hlnum{35}\hldef{,}\hlnum{45}\hldef{,}\hlnum{55}\hldef{,}\hlnum{65}\hldef{)}
\hldef{x}
\end{alltt}
\begin{verbatim}
## [1] 15 25 35 45 55 65
\end{verbatim}
\begin{alltt}
\hldef{y} \hlkwb{<-} \hlkwd{c}\hldef{(}\hlnum{302.38}\hldef{,} \hlnum{193.63}\hldef{,} \hlnum{185.46}\hldef{,} \hlnum{198.49}\hldef{,} \hlnum{224.30}\hldef{,} \hlnum{288.71}\hldef{)}
\hldef{y}
\end{alltt}
\begin{verbatim}
## [1] 302.38 193.63 185.46 198.49 224.30 288.71
\end{verbatim}
\begin{alltt}
\hlcom{# Scatter plot}
\hlkwd{plot}\hldef{(x,y,} \hlkwc{main}\hldef{=}\hlsng{"Average age vs. time spent in the library"}\hldef{,}
     \hlkwc{xlab}\hldef{=}\hlsng{"Age"}\hldef{,} \hlkwc{ylab}\hldef{=}\hlsng{"Time spent in the library"}\hldef{,}\hlkwc{col}\hldef{=}\hlsng{"red"}\hldef{)}
\end{alltt}
\end{kframe}
\includegraphics[width=\maxwidth]{figure/unnamed-chunk-1-1} 
\end{knitrout}
\subsection{Manual method}
Problem: \newline
Find the correlation coefficient for the given data: x=(23,27,28,28,29,30,31,32, 33,35), y=(18,20,22,27,21,29,27,29,28,29). Give a scatter plot to the x and y
\begin{knitrout}
\definecolor{shadecolor}{rgb}{0.969, 0.969, 0.969}\color{fgcolor}\begin{kframe}
\begin{alltt}
\hldef{x}\hlkwb{=}\hlkwd{c}\hldef{(}\hlnum{23}\hldef{,}\hlnum{27}\hldef{,}\hlnum{28}\hldef{,}\hlnum{28}\hldef{,}\hlnum{29}\hldef{,}\hlnum{30}\hldef{,}\hlnum{31}\hldef{,}\hlnum{32}\hldef{,} \hlnum{33}\hldef{,}\hlnum{35}\hldef{)}
\hldef{x}
\end{alltt}
\begin{verbatim}
##  [1] 23 27 28 28 29 30 31 32 33 35
\end{verbatim}
\begin{alltt}
\hlkwd{length}\hldef{(x)}
\end{alltt}
\begin{verbatim}
## [1] 10
\end{verbatim}
\begin{alltt}
\hldef{y}\hlkwb{=}\hlkwd{c}\hldef{(}\hlnum{18}\hldef{,}\hlnum{20}\hldef{,}\hlnum{22}\hldef{,}\hlnum{27}\hldef{,}\hlnum{21}\hldef{,}\hlnum{29}\hldef{,}\hlnum{27}\hldef{,}\hlnum{29}\hldef{,}\hlnum{28}\hldef{,}\hlnum{29}\hldef{)}
\hldef{y}
\end{alltt}
\begin{verbatim}
##  [1] 18 20 22 27 21 29 27 29 28 29
\end{verbatim}
\begin{alltt}
\hlkwd{length}\hldef{(y)}
\end{alltt}
\begin{verbatim}
## [1] 10
\end{verbatim}
\begin{alltt}
\hlkwd{var}\hldef{(x)}
\end{alltt}
\begin{verbatim}
## [1] 11.6
\end{verbatim}
\begin{alltt}
\hlkwd{var}\hldef{(y)}
\end{alltt}
\begin{verbatim}
## [1] 18.22222
\end{verbatim}
\begin{alltt}
\hlkwd{var}\hldef{(x, y)}
\end{alltt}
\begin{verbatim}
## [1] 12.11111
\end{verbatim}
\begin{alltt}
\hldef{r}\hlkwb{=}\hlkwd{var}\hldef{(x,y)}\hlopt{/}\hlkwd{sqrt}\hldef{(}\hlkwd{var}\hldef{(x)}\hlopt{*}\hlkwd{var}\hldef{(y))}
\hldef{r}
\end{alltt}
\begin{verbatim}
## [1] 0.8330179
\end{verbatim}
\end{kframe}
\end{knitrout}
Therfore there is positive correlation between the x and y.
% <<>>=
% plot(x,y, main="Scatter plot", 
%      xlab="X", ylab="Y",col="blue")
% @
\subsection{Using R code}
\subsubsection{Method 1}
\begin{knitrout}
\definecolor{shadecolor}{rgb}{0.969, 0.969, 0.969}\color{fgcolor}\begin{kframe}
\begin{alltt}
\hlkwd{cor}\hldef{(x,y,} \hlkwc{method} \hldef{=}\hlsng{"pearson"} \hldef{)}
\end{alltt}
\begin{verbatim}
## [1] 0.8330179
\end{verbatim}
\end{kframe}
\end{knitrout}
\subsubsection{Method 2}
\begin{knitrout}
\definecolor{shadecolor}{rgb}{0.969, 0.969, 0.969}\color{fgcolor}\begin{kframe}
\begin{alltt}
\hlkwd{cor.test}\hldef{(x,y,}\hlkwc{method}\hldef{=}\hlsng{"pearson"}\hldef{)}
\end{alltt}
\begin{verbatim}
## 
## 	Pearson's product-moment correlation
## 
## data:  x and y
## t = 4.2587, df = 8, p-value = 0.002766
## alternative hypothesis: true correlation is not equal to 0
## 95 percent confidence interval:
##  0.4277335 0.9594318
## sample estimates:
##       cor 
## 0.8330179
\end{verbatim}
\end{kframe}
\end{knitrout}
\section{Spearman's Rank Correlation Coefficient}
Problem : \newline 
Twelve recruits were subjected to selection test to ascertain their
suitability for a certain course of training. At the end of training they were given a
proficiency test. The marks scored by the recruits are recorded below
\begin{table}[!ht]
    \centering
    \begin{tabular}{|c|c|c|c|c|c|c|c|c|c|}
    \hline
        Recruit & 1 & 2 & 3 & 4 & 5 & 6 & 7 & 8 & 9 \\ \hline
        Selection Test Score & 44 & 49 & 52 & 54 & 47 & 76 & 65 & 60 & 63 \\ \hline
        Proficiency Test Scrore & 48 & 55 & 45 & 60 & 43 & 80 & 58 & 50 & 77 \\ \hline
       
    \end{tabular}
\end{table}
\begin{knitrout}
\definecolor{shadecolor}{rgb}{0.969, 0.969, 0.969}\color{fgcolor}\begin{kframe}
\begin{alltt}
\hldef{selection} \hlkwb{=}\hlkwd{c}\hldef{(}\hlnum{44}\hldef{,}\hlnum{49}\hldef{,}\hlnum{52}\hldef{,}\hlnum{54}\hldef{,}\hlnum{47}\hldef{,}\hlnum{76}\hldef{,}\hlnum{65}\hldef{,}\hlnum{60}\hldef{,}\hlnum{63}\hldef{,}\hlnum{58}\hldef{,}\hlnum{50}\hldef{,}\hlnum{67}\hldef{)}
\hldef{proficiency} \hlkwb{=}\hlkwd{c}\hldef{(}\hlnum{48}\hldef{,}\hlnum{55}\hldef{,}\hlnum{45}\hldef{,}\hlnum{60}\hldef{,}\hlnum{43}\hldef{,}\hlnum{80}\hldef{,}\hlnum{58}\hldef{,}\hlnum{50}\hldef{,}\hlnum{77}\hldef{,}\hlnum{46}\hldef{,}\hlnum{47}\hldef{,}\hlnum{65}\hldef{)}
\hlkwd{cor}\hldef{(selection,proficiency,}\hlkwc{method} \hldef{=}\hlsng{"spearman"}\hldef{)}
\end{alltt}
\begin{verbatim}
## [1] 0.7202797
\end{verbatim}
\begin{alltt}
\hlcom{# (or)}
\hlkwd{cor.test}\hldef{(selection,proficiency,}\hlkwc{method} \hldef{=}\hlsng{"spearman"}\hldef{)}
\end{alltt}
\begin{verbatim}
## 
## 	Spearman's rank correlation rho
## 
## data:  selection and proficiency
## S = 80, p-value = 0.01102
## alternative hypothesis: true rho is not equal to 0
## sample estimates:
##       rho 
## 0.7202797
\end{verbatim}
\end{kframe}
\end{knitrout}
There is a positive correlation between selection and proficiency.
\\ \\ \\
Problem : \newline 
Import the iris data set. \\
i) Find the Pearson correlation coefficient using the manual method for the features sepal length and sepal width when the Species is setosa. Give a plot visualization. Describe the relationship.\\
ii) Find the rank correlation coefficient for the features petal length and petal width when the Species is versicolor and describe the relationship.\\
iii) Find the Pearson correlation coefficient for the features petal length and petal width when the Species is virginica and describe the relationship.
\begin{knitrout}
\definecolor{shadecolor}{rgb}{0.969, 0.969, 0.969}\color{fgcolor}\begin{kframe}
\begin{alltt}
\hldef{data} \hlkwb{<-} \hldef{iris}
\hlkwd{head}\hldef{(data)}
\end{alltt}
\begin{verbatim}
##   Sepal.Length Sepal.Width Petal.Length Petal.Width Species
## 1          5.1         3.5          1.4         0.2  setosa
## 2          4.9         3.0          1.4         0.2  setosa
## 3          4.7         3.2          1.3         0.2  setosa
## 4          4.6         3.1          1.5         0.2  setosa
## 5          5.0         3.6          1.4         0.2  setosa
## 6          5.4         3.9          1.7         0.4  setosa
\end{verbatim}
\begin{alltt}
\hldef{setosa_data} \hlkwb{<-} \hlkwd{subset}\hldef{(data, Species} \hlopt{==} \hlsng{"setosa"}\hldef{)}

\hldef{r}\hlkwb{=}\hlkwd{var}\hldef{(setosa_data}\hlopt{$}\hldef{Sepal.Length, setosa_data}\hlopt{$}\hldef{Sepal.Width}
      \hldef{)}\hlopt{/}\hlkwd{sqrt}\hldef{(}\hlkwd{var}\hldef{(setosa_data}\hlopt{$}\hldef{Sepal.Length)}\hlopt{*}
               \hlkwd{var}\hldef{(setosa_data}\hlopt{$}\hldef{Sepal.Width))}
\hldef{r}
\end{alltt}
\begin{verbatim}
## [1] 0.7425467
\end{verbatim}
\end{kframe}
\end{knitrout}
\begin{knitrout}
\definecolor{shadecolor}{rgb}{0.969, 0.969, 0.969}\color{fgcolor}\begin{kframe}
\begin{alltt}
\hlkwd{plot}\hldef{(setosa_data}\hlopt{$}\hldef{Sepal.Length, setosa_data}\hlopt{$}\hldef{Sepal.Width,}
     \hlkwc{main} \hldef{=} \hlsng{"Length vs Width"}\hldef{,}  \hlkwc{xlab} \hldef{=}\hlsng{"Sepal Length"}\hldef{,}
     \hlkwc{ylab} \hldef{=} \hlsng{"Sepal Width"}\hldef{,} \hlkwc{col}\hldef{=} \hlsng{"green"}\hldef{)}
\end{alltt}
\end{kframe}
\includegraphics[width=\maxwidth]{figure/unnamed-chunk-7-1} 
\end{knitrout}
There is a positive correlation between sepal length and sepal width.
\begin{knitrout}
\definecolor{shadecolor}{rgb}{0.969, 0.969, 0.969}\color{fgcolor}\begin{kframe}
\begin{alltt}
\hldef{versi_data} \hlkwb{<-} \hlkwd{subset}\hldef{(data, Species} \hlopt{==} \hlsng{"versicolor"}\hldef{)}
\hlkwd{cor}\hldef{(versi_data}\hlopt{$}\hldef{Petal.Length, versi_data}\hlopt{$}\hldef{Petal.Width,}
         \hlkwc{method} \hldef{=} \hlsng{"spearman"}\hldef{)}
\end{alltt}
\begin{verbatim}
## [1] 0.7870096
\end{verbatim}
\end{kframe}
\end{knitrout}
There is a positive correlation between petal length and petal width.
\begin{knitrout}
\definecolor{shadecolor}{rgb}{0.969, 0.969, 0.969}\color{fgcolor}\begin{kframe}
\begin{alltt}
\hldef{virginica_data} \hlkwb{<-} \hlkwd{subset}\hldef{(data, Species} \hlopt{==} \hlsng{"virginica"}\hldef{)}
\hlkwd{cor}\hldef{(virginica_data}\hlopt{$}\hldef{Petal.Length, virginica_data}\hlopt{$}\hldef{Petal.Width,}
         \hlkwc{method} \hldef{=} \hlsng{"pearson"}\hldef{)}
\end{alltt}
\begin{verbatim}
## [1] 0.3221082
\end{verbatim}
\end{kframe}
\end{knitrout}
There is a positive correlation between petal length and petal width.\\ \\

\section*{Experiment}
1. Import the Orange data set.
Find the Pearson correlation coefficient for each type of tree age and circumference. Also draw the scatter plot.\\
2. Import the trees data set. \\
i) Find the rank correlation for Girth  and Volume. Draw a plot.\\
ii) Find the rank correlation for Height  and Volume. Draw a plot.






\end{document}
